\documentclass[dvipdfmx]{jsarticle}

\title{カレンダーをつくる(サーブレット版)}
\author{Seiichi Nukayama}
\date{2020-07-15}
\usepackage{tcolorbox}
\usepackage{color}
\usepackage{listings, plistings}

% Java
\lstset{% 
  frame=single,
  backgroundcolor={\color[gray]{.9}},
  stringstyle={\ttfamily \color[rgb]{0,0,1}},
  commentstyle={\itshape \color[cmyk]{1,0,1,0}},
  identifierstyle={\ttfamily}, 
  keywordstyle={\ttfamily \color[cmyk]{0,1,0,0}},
  basicstyle={\ttfamily},
  breaklines=true,
  xleftmargin=0zw,
  xrightmargin=0zw,
  framerule=.2pt,
  columns=[l]{fullflexible},
  numbers=left,
  stepnumber=1,
  numberstyle={\scriptsize},
  numbersep=1em,
  language={Java},
  lineskip=-0.5zw,
  morecomment={[s][{\color[cmyk]{1,0,0,0}}]{/**}{*/}},
}
%\usepackage[dvipdfmx]{graphicx}
\usepackage{url}
\usepackage[dvipdfmx]{hyperref}
\usepackage{amsmath, amssymb}
\usepackage{itembkbx}
\usepackage{eclbkbox}	% required for `\breakbox' (yatex added)
\fboxrule=1pt
\parindent=1em
\begin{document}

%% 修正時刻: Wed Jul 15 14:49:27 2020


\section{Calendarクラス}

当月のカレンダーを表示するサーブレットを作成する。

\begin{tcolorbox}
Calendar.class \\
java.uti.Calendar
\end{tcolorbox}

Java の Calendarクラスは以下のようにして使う。

抽象クラスなのでインスタンスの生成はできない。

かわりに、getInstance()メソッドをもっていて、これで
カレンダーオブジェクトを作成できる。

\vspace{3mm}
\fbox{Calendar calendar = Calendar.getInstance();}
\vspace{3mm}

この calendarオブジェクトは、現在の日時で初期化されている。

ここまでをプログラムしてみる。

\begin{lstlisting}[caption=MyCalendar.java]
import java.util.Calendar;

class MyCalendar {
  public static void main (String[] args) {

    Calendar calendar = Calendar.getInstance();
  }
}
\end{lstlisting}


これで作成した calendarオブジェクトはいろいろ便利なメソッドを
もっている。
\vspace{3mm}

\fbox{(int) calendar.get( Calendar.YEAR )}

年が返る。今回の場合、現在日時で初期化されているから、現在の年が整数型で取得できる。
\vspace{3mm}

\fbox{(int) calendar.get( Calendar.MONTH )}

月が返る。現在の月が整数で取得できる。ただし、1月(January)は 0 である。12月は 11 となる。
\vspace{3mm}

\fbox{(int) calendar.get( Calendar.DATE )}

日が返る。現在の日付が取得できる。
\vspace{3mm}

\fbox{(int) calendar.get( Calendar.DAY\_OF\_WEEK )}

曜日が返る。日曜日が 1 である。土曜日は 7 となる。
\vspace{3mm}

ここまでをプログラムしてみる。

\begin{lstlisting}[caption=MyCalendar.java]
    import java.util.Calendar;

class MyCalendar {
    public static void main (String args[] ) {
        Calendar calendar = Calendar.getInstance();

        System.out.println(calendar.get(Calendar.YEAR));    // 2020

        System.out.println(calendar.get(Calendar.MONTH));   // 6

        System.out.println(calendar.get(Calendar.DATE));    // 今日の日付

        System.out.println(calendar.get(Calendar.DAY_OF_WEEK));   // 4 すなわち水曜日
    }
}
\end{lstlisting}

以下のようにコンパイルして動作させてみる。

\begin{tcolorbox}
\begin{verbatim}
 > javac MyCalendar.java
 > java MyCalendar
 2020
 6
 15
 4
\end{verbatim}
\end{tcolorbox}

現在の日付によって出力された数字は違うが、年、月、日、曜日が数字で
出力されるはずである。

また、次のように日を指定して、その日の情報を取得することもできる。

\begin{lstlisting}
 calendar.set( 2020, 6, 1 );     // 7月1日をセット
 System.out.println( calendar.get( Calendar.DAY_OF_WEEk ));   // 4
\end{lstlisting}

\textbf{4} すなわち ''水曜日'' と出力される。

また、前の月の最終日が何日だったかは、次のようにすればわかる。

\begin{lstlisting}
 calendar.set( 2020, 6, 0 );    // 7月1日の前日をセット
 System.out.println( calendar.get( Calendar.DATE ));    // 30
\end{lstlisting}

\section{カレンダーを作成する}

\subsection{Eclipseの場合}

カレンダーアプリを作成する。\\
Eclipseで作業している場合は、新規プロジェクトで''動的Webプロジェクト'' を選び、プロジェクト名を ''calendar'' とする。\\
ターゲットランタイムは、''Apache Tomcat v9.0''、動的Webモジュールバージョンは ''4.0''。あとはそのままでよい。

Eclipseのエディター画面が開いたら、''新規'' ー ''サーブレット'' と選択し、
サーブレット作成画面で、''Javaパッケージ''には ''\textgt{servlet}''、クラス名には ''\textgt{MyCalendar}'' と入力する。\\
次に開いた ''初期化パラメーター'' と ''URLマッピング'' は、そのままでよい。\\
''どのメソッドスタブを作成しますか?'' には、スーパークラスも使わないし、postも使わないので、その二つのチェックははずしておいてかまわない。(別につけたままで、あとからそのメソッドを削除してもよい)


\subsection{コマンドラインで作業している場合}

適当なフォルダを作成し、その名前を ''calendar'' とする。\\
説明の都合上、''\verb!C:\Users\user\Documents\calendar!'' とする。
Tomcatがインストールされているフォルダを ''\verb!C:\pleiades\tomcat\9!'' だとすると、
''\verb!C:\pleiades\tomcat\9\conf\Catalina\localhost!'' フォルダに ''calendar.xml'' を作成する。

\begin{lstlisting}[caption=calendar.xml]
 <?xml version='1.0' encoding='utf-8'?>
<Context path="/calendar" docBase="C:/Users/user/Documents/calendar" />
\end{lstlisting}

また、作成した ''calendar''フォルダを以下のようにする。

\begin{verbatim}
./calendar/
  ├── WEB-INF/
  │   └── classes/
  └── src/
      └── servlet/
          └── MyCalendar.java
\end{verbatim}

上のように src/servlet フォルダに ''MyCalendar.java'' を作成しておく。

その内容は、以下である。

\begin{lstlisting}
package servlet;

import java.io.IOException;
import java.io.PrintWriter;

import javax.servlet.ServletException;
import javax.servlet.annotation.WebServlet;
import javax.servlet.http.HttpServlet;
import javax.servlet.http.HttpServletRequest;
import javax.servlet.http.HttpServletResponse;

import java.util.Calendar;

@WebServlet("/MyCalendar")
public class MyCalendar extends HttpServlet {
  private static final long serialVersionUID = 1L;

  // ここにクラス変数を書く
 
  protected void doGet( HttpServletRequest request,
                       HttpServletResponse response)
               throws ServletException, IOException {

    // ここに 処理を書く
 
  }
}
\end{lstlisting}

\section{どのようなプログラムにするかを考える}

\subsection{できあがりのイメージ}

できあがりのイメージは以下である。

\begin{tcolorbox}
\begin{verbatim}
      2020年 7月
  日 月 火 水 木 金 土
  28 29 30  1  2  3  4
   5  6  7  8  9 10 11
  12 13 14 15 16 17 18
  19 20 21 22 23 24 25
  26 27 28 29 30 31  1

  本日は 2020年7月15日
\end{verbatim}
\end{tcolorbox}

上のタイトル部分の ``2020''で使う変数を \textbf{year} 、''7'' を \textbf{month} とする。

カレンダー部分の数字は、配列を使うことにする。変数名は \textbf{calendarDay[]} とする。

配列の中の添字部分は、\textbf{count} という変数とする。

また、該当月が何週あるかを \textbf{weekCount} という変数とする。

一番下の本日の日付を表示する部分だが、''15''で使う変数を \textbf{day} とする。

\subsection{プログラムの構成}

クラスの先頭部分でクラス変数を定義する。\\
year, month, day, calendarDay[], count, weekCount である。

doGet()メソッドで、ブラウザへの処理をおこなう。

getCalendar()メソッドを作成し、カレンダー計算などの処理はここで行う。
クラス変数の内容もここで決まる。


\begin{lstlisting}
 class Calendar {

   +----------------+
   | クラス変数定義 |
   +----------------+

   protected void doGet(...) {
     +----------------+
     |  画面処理      |
     | クラス変数出力 |
     +----------------+
   }
 
   private void getCalendar () {
     +------------------+
     |  カレンダー計算  |
     |  クラス変数内容  |
     +------------------+
   }
 }
\end{lstlisting}

\section{プログラムリスト}

プログラムリストは以下である。

\begin{lstlisting}
package servlet;

import java.io.IOException;
import java.io.PrintWriter;

import javax.servlet.ServletException;
import javax.servlet.annotation.WebServlet;
import javax.servlet.http.HttpServlet;
import javax.servlet.http.HttpServletRequest;
import javax.servlet.http.HttpServletResponse;
import java.util.Calendar;

@WebServlet("/MyCalendar")
public class MyCalendar extends HttpServlet {
  private static final long serialVersionUID = 1L;
	
  private int year;
  private int month;
  private int day;
  	
  private int [] calendarDay;
  private int count;
  private int weekCount;
  	
  protected void doGet( HttpServletRequest request,
                        HttpServletResponse response)
                 throws ServletException, IOException {

    response.setContentType("text/html; charset=UTF-8");
    PrintWriter out = response.getWriter();
    // 今月のカレンダー情報を取得
    getCalendar();
    
    String todayStr =
    	String.valueOf(year) + "年" +
    	String.valueOf(month + 1) + "月" +
    	String.valueOf(day) + "日";
    	
    String dateTimeStr = 
    	String.valueOf(year) + "-" +
    	String.valueOf(month) + "-" +
    	String.valueOf(day);
    		
    // 表示データを出力
    out.println("<!doctype html>");
    out.println("<html lang=\"ja\">");
    out.println("<head>");
    out.println("<meta charset=\"utf-8\">");
    out.println("<link rel=\"stylesheet\" href=\"calendar.css\">");
    out.println("</head>");
    out.println("<body>");
    out.println("<h1>" + year + "年" + (month + 1) + "月</h1>") ;
    out.println("<table>");
    out.println("<tr>");
    out.println("<th>日</th>");
    out.println("<th>月</th>");
    out.println("<th>火</th>");
    out.println("<th>水</th>");
    out.println("<th>木</th>");
    out.println("<th>金</th>");
    out.println("<th>土</th>");
    out.println("</tr>");
    out.println("<tr>");
    		
    for (int i = 0; i < weekCount; i++) {
      for (int j = i * 7; j < i * 7 + 7; j++) {
    	if (j % 7 == 0) {
    	  out.println("<td class=\"sun\">" + calendarDay[j] + "</td>");
    	} else if (j % 7 == 6) {
    	  out.println("<td class=\"sat\">" + calendarDay[j] + "</td>");
    	} else {
    	  out.println("<td>" + calendarDay[j] + "</td>");
    	}
      }
      out.println("</tr><tr>");
    }
    		
    out.println("</tr>");
    out.println("</table>");
    out.println("<p>今日は <time datetime=" + dateTimeStr + ">" + todayStr + "</time></p>");
    out.println("</body>");
    out.println("</html>");
  }
    	
  private void getCalendar ( ) {
    // getInstance() -- 現在の日時でのカレンダーを取得する
    Calendar calendar = Calendar.getInstance();
    
    year = calendar.get(Calendar.YEAR);
    month = calendar.get(Calendar.MONTH);
    day = calendar.get(Calendar.DATE);
    
    // 今月が何曜日から開始されているか?
    // 今月の1日をセット
    calendar.set( year, month, 1 );
    // 1日の曜日を取得 -- Sunday:1 ... Saturday:7 
    int startWeek = calendar.get( Calendar.DAY_OF_WEEK);
    
    // 先月が何日までだったか?
    // 今月の1日の前日をセット
    calendar.set( year, month, 0 );
    // 先月の最終の日付を取得
    int beforeMonthLastDay = calendar.get( Calendar.DATE );
    
    // 今月が何日までか?
    // 来月の1日の前日をセット
    calendar.set( year, month + 1, 0 );
    // 今月の最終の日付を取得
    int thisMonthLastDay = calendar.get( Calendar.DATE );
    
    // カレンダーの各欄を配列で用意
    calendarDay = new int[42];
    // カレンダーの各欄の通し番号
    count = 0;
    
    // 先月分の日付をセット
    // startWeekが月曜日つまり 2 ならば、i=0 となる。
    // calendarDay[0] には先月の最終日の日付がセットされる。
    for (int i = startWeek - 2; i >= 0; i--) {
      calendarDay[count] = beforeMonthLastDay - i;
      count++;
    }
    
    // 今月分の日付をセット
    for (int i = 1; i <= thisMonthLastDay; i++) {
      calendarDay[count] = i;
      count++;
    }
    
    // 翌月分の日付をセット
    int nextMonthDay = 1;
    // カレンダーの欄に余裕がある限り
    while (count % 7 != 0) {
      calendarDay[count] = nextMonthDay++;
      count++;
    }
    
    // カレンダーに何週分あるか
    weekCount = count / 7;
  }
  
}
\end{lstlisting}

\subsection{カレンダーのデザイン}

カレンダーに罫線を引いたりするので、スタイルシートを設置する。\\
''calendar''フォルダの直下に、\textbf{calendar.css} というファイル名で作成する。

\begin{lstlisting}[caption=calendar.css]
 @charset "utf-8";

table, th, td {
	border: solid 1px #444;
}

table {
	border-collapse: collapse;
}

td {
	text-align: center;
}

.sun {
	background-color: pink;
}

.sat {
	background-color: #a2e6ef;
}
\end{lstlisting}

\section{サーバーの再起動}

サーブレットの場合はサーバーを再起動しなくてはならない。

\subsection{Eclipseの場合}

サーバーに Calenderプロジェクトが追加されていることを確認して、再起動する。

そののちに \textbf{http://localhost:8080/calendar/MyCaledar} にアクセスする。

\subsection{コマンドラインの場合}

calendarフォルダ直下にて、以下のようにコンパイルを行う。

\begin{tcolorbox}
 javac -d ./WEB-INF/classes src/servlet/*.java
\end{tcolorbox}

Tomcatフォルダの中の binフォルダの中の startup.bat をダブルクリックして起動させる。

\textbf{http://localhost:8080/calendar/MyCalendar} にアクセスするとカレンダーが表示されるはずである。




\end{document}

%% 修正時刻: Sat May  2 15:10:04 2020


%% 修正時刻: Wed Jul 15 19:43:02 2020
